\section{Building a R\'esum\'e}
Return to your main file \emph{CVexample.tex}. \par

Now, you should be able to easily identify where the \lstinline|\tab| command has been used. \par

In each case, we can modify the arguments passed to the command to fill out the contents of our r\'esum\'e. We'll do this section-by-section. \par

\subsection{Educational History}
In most cases, it is appropriate to make the first section of your r\'esum\'e about your educational history; unless you feel that your work experience is particularly relevant for the position you're applying to. \par

Step one is to rewrite the first \lstinline|\tab| command, with your most recent qualification. (This will most likely to be your last, or current degree program). \par

\begin{instrct}
Find the first instance of the \lstinline|\tab| command in \emph{CVexample.tex} and rewrite the four arguments. 
\end{instrct}

Remember that in your main document, the \lstinline|tab| command is formatted as follows: \par

\begin{lstlisting}
1.  \tab %
2.    {#1}{#2}
3.    {#3}{#4}
\end{lstlisting}

The first line (corresponding to arguments \lstinline|#1| and \lstinline|#2|) will be \textbf{bold and larger}. \par

The second line (arguments \lstinline|#3| and \lstinline|#4|) will be {\footnotesize{\textit{\sffamily smaller, italicized and sans-serif}}}. At first glance, this text will be less noticable. \par

Consider then, the best way to arrange this information. What should be prioritized for an employer to see? \par

You probably want your degree program to stand out. This should be the first argument, next to your period of study. \par

On the second line, you could put the name of your awarding body (i.e. university) and optionally, the city, and country that you studied in. \par

\clearpage

Directly below the first \lstinline|\tab| command is the following element: \par

\begin{lstlisting}
4.  \begin{itemize}
5.    \item
6.  \end{itemize}
\end{lstlisting}

Which creates a bulleted list environment. \par

The list environment is contained within the expressions \lstinline|\begin{itemize}| and \lstinline|\end{itemize}|. A bullet point is created with the expression \lstinline|\item|, followed by plain text. \par

This is an optional space to add more detail about your qualification. \par

You may choose to give specific examples of your skills, projects that you worked on, significant achievements or classes that you attended. \par

\begin{instrct}
Within the list environment, add some more information to your r\'esum\'e.
\end{instrct}

You likely have multiple qualifications that you'd like to show off. \par

Luckily, the same block of code can be copied-and-pasted to create another subheading. \par

\begin{ext}
Using \lstinline|\tab| and a list environment, create another subheading. Rinse and repeat for all your qualifications. 
\end{ext}

\subsection{Technical Skills}
In most cases, your work experience should follow your qualifications. Unless your work experience isn't particularly relevant for the role you're applying to. \par

I.e. students with part-time work experience that are applying for laboratory research positions. \emph{Sound familiar}? \par

So instead, try focusing on your industry-specific experience and technical skills; which employers will want to see evidence of. \par

For example, a lab assisstant position may ask for evidence of your ability to carry out titrations, perform western blot or qPCR analyses. \par

\clearpage

Consider the position that you are interested in. What skills are the employer looking for? \par

It is useful to research the position that you want whilst, or even before writing your application. Different employers will have different expectations and you should try tailoring your r\'esum\'e to the job. \par

With that being said, don't be dissauded from applying for a position you may be underqualified for. Sometimes this is not a deal-breaker. \par

\begin{instrct}
Make a list of skills that your employer is looking for. Which of these do you already have? Are they included in your r\'esume\'?
\end{instrct}


\emph{CVexample.tex} contains two examples of subheadings, underneath the section \emph{Technical Skills}. Notice that the right-justifed cells have been left empty, because we don't need to include certain information. \par


