\section{Working with \TeX~Files}
In most cases, PDFs cannot be directly edited. Instead the original \TeX~file must be recompiled with changes; the corresponding PDF is then updated to match. \par

The following sections will explain how to navigate and edit \TeX~files. \par

\subsection{Parts of a Document}
Returning to \lstinline|CVexample.tex|, can you identify structural similarities between your \TeX~file and the template? \par

\TeX~files always begin with a class declaration, which determines the kind of document that is produced. The expression \lstinline|\documentclass{articleCV}| loads the file \lstinline|articleCV.cls|, which is an example of an \emph{inherited class} (more on this later). \par

The `body', or contents of a text document are contained between the declarations \lstinline|\begin{document}| and \lstinline|\end{document}|. This includes all the text that is output when the document is compiled. \par

Like the template, the \TeX~file contains three sections, which are delimited by the command \lstinline|\section{}| and followed by elements begining with the command \lstinline|\tab{}|. This is a custom command and one we'll return to later. For now, remember that these elements format the body of your r\'esum\'e. \par

Near the top of your TeX file is the code that will format your document's header, which includes two blocks of code. Although it may look intimidating, this is where we'll start editing. \par

\subsection{The Header pt.1}
The first part of the document header describes the title, R\'esum\'e/CV and the author's name. It looks like: \par

\begin{lstlisting}
25.  \begin{centering}
26.    {\large \scshape R\'esum\'e/CV \par}
27.    {\Huge Pedro Henrique Mattos \par}
28.  \end{centering}
\end{lstlisting}

When a \TeX~document is compiled, only plain text is output. Special characters, e.g. \lstinline|{}| or commands (prepended by the \lstinline|\| symbol) are ignored --- as they control how your document looks. \par

So, we can safely change the contents and worry about the formatting later. \par

\begin{instrct}
Rewrite the title and replace my name with your own. Then, recompile your document.
\end{instrct}

Now, we can consider how these lines are formatted. \par

The environment created by \lstinline|25.\begin{centering}| and \lstinline|28.\end{centering}| center the title on the page. Removing these lines would automtically left-adjust the text, which may be preferred. \par

Lines 26. and 27. contain the text that is output when the document is compiled.

\begin{lstlisting}
26.    {\large \scshape R\'esum\'e/CV \par}
27.    {\Huge Pedro Henrique Mattos \par}
\end{lstlisting}

Notice that the title and the author are styled differently. Each line is enclosed by a pair of braces, so that the styles don't overlap. \par

The command \lstinline|\large| and \lstinline|\scshape| increase the font {\large size} and set {\scshape small caps}, respectively. Compare this with the following line, where the font size is set to \lstinline|\Huge|. \par

\LaTeX~has many different font styles, which are worth exploring. \par

\clearpage 

\subsection{The Header pt.2}
The second part of the header looks more complicated. The output file contains a table with two columns, but these aren't immediately obvious within our TeX file. \par

\begin{lstlisting}
32.  \begin{table}[ht]
33.    \centering
34.    \begin{tabular}{>{\small} l >{\footnotesize} l}
35.    <Undergraduate, 3rd Year>, & Email: \Email{\email} \\
36.    \dept, & Tel: \phone \\
37.    \org & \url{\website}
38.    \end{tabular}
39.  \end{table}
\end{lstlisting}

Fortunately, we can once again edit the contents of our header, before worrying about the formatting. The following three lines describe the contents of the table. \par

\begin{lstlisting}
35.    <Undergraduate, 3rd Year>, & Email: \Email{\email} \\
36.    \dept, & Tel: \phone \\
37.    \org & \url{\website}
\end{lstlisting}

Each line describes a single row of the table. Separations between columns, and linebreaks are indicated by the ampersand \lstinline|&| and double backslash \lstinline|\\| symbols, respectively. \par

Excluding \lstinline|<Undergraduate, 3rd Year>|, each entry of the table is actually a custom command that references your department , organisation, email, phone number and website. \par

These commands are defined within the file \lstinline|.sty/info.sty|; by rewriting them, we can change what is output in the table. This is a useful trick for describing information that is often repeated and can be applied across multiple files. \par

\begin{instrct}
Navigate to the file \lstinline|.sty/info.sty| in your project directory.
\end{instrct}

Commands are defined as \lstinline|\newcommand{\<tag>}{<definition>}|. The definition is output whenever a command is referenced by its tag. \par

\begin{instrct}
Rewrite the commands within \lstinline|.sty/info.sty|. Then, save your changes and recompile \lstinline|CVexample.tex|.
\end{instrct}

Don't forget to also change the expression \lstinline|<Undergraduate, 3rd Year>| with your own title or position, in the case that you are not a 3rd year undergraduate student. \par

Now, let's look at the construction of the table. (Skip ahead to the next section if this feels like it's above your pay-grade). \par

The commands \lstinline|34.\begin{tabular}| and \lstinline|38.\end{tabular}| initialise a \emph{tabular} environment, which produces a table. Confusingly, the environment created by \lstinline|32.\begin{table}| and \lstinline|39.\end{table}| is a float, which controls the position of the table on the page. \par

Look at \lstinline|34.\begin{tabular}{>{\small} l >{\footnotesize} l}|, which initializes the table. \par

The letter \lstinline|l|, repeated twice, indicated that the table contains two left-adjusted columns. Then, \lstinline|>{\small}| and \lstinline|>{\footnotesize}| prepend the respective styling to either column. So, text is {\small small} within the first column and {\footnotesize footnotesize} in the second. \par

Lastly, \lstinline|33.\centering| centers everything within the \emph{table} environment on the page. \par

%\subsection{A Note on Document Formatting}
%The main body of a r\'esum\'e document is generally limited to short `sections' or subheadings, that normally include details about a relevant position, qualification or experience and optionally a date; an organisation and a location. Often, these subheadings are followed by bullet-points providing more information. \par
%
%In the PDF document, subheadings are elements of two lines, with text justified to both the right and left margins. Text in the first line is larger, and bold. Text in the second line is smaller, sans-serif and italicized. \par
%
%\subsection{Creating R\'esum\'e Subheadings}
%The class file \emph{articleCV.cls} contains definitions for a custom command \lstinline|\tab| that creates a subheading with this structure. These require loading two packages \emph{tabularx} and {array}. \par
%
%The \lstinline|\tab{}| command is defined in the following code block: \par
%
%\begin{lstlisting}
%1.  \newcommand{\tab}[4]{
%2.    \begin{tabularx}{1\textwidth}{@{} $ X ^ r}
%3.    \rowstyle{\bfseries} {#1} & {#2} \\
%4.    \rowstyle{\itshape\sffamily\small} {#3} & {#4} \\
%5.    \end{tabularx}
%6.  }
%\end{lstlisting}
%
%\lstinline|6.\newcommand{\tab}[4]{| creates a new command, with the name \lstinline|\tab|, that accepts four arguments, which can be NULL. Arguments to a command are supplied within curly braces, which must be present.
%
%Thus in the main file, a header is created by writing the command: \par
%
%\begin{lstlisting}
%  \tab %
%    {#1}{#2}
%    {#3}{#4}
%\end{lstlisting}
%
%Excecution of the \lstinline|\tab| command simply creates a table, with two columns and two rows, whose arguments are given as inputs. \par
%
%Formatting of each row is controlled by \lstinline|3.\rowstyle{\bfseries}| and \lstinline|4.\rowstyle{\itshape\sffamily\small|, which respectively make the first row bold, and the second row italicized, sans-serif and small. Changing these parameters will change the style of your r\'esum\'e's subsections. \par
%
%For more detail on how the \lstinline|\tab{}| command was implemented, see section~5.1. \par