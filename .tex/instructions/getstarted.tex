\section{Get Started}
%\subsection{Usage}
%\begin{instrct}
%This is an instruction.
%\end{instrct}
%
%Ocassionally, an extension task will prompt you with something more complicated. Don't worry, these are optional. \par
%
%\begin{ext}
%This is an optional task.
%\end{ext}

%\setcounter{step}{0}

If you haven't already, begin by following the instructions located at \lstinline|Pedro-h-mattos/LaTeXCVTemplate|. \par

They instruct you to
\begin{enumerate*}[label=(\roman*)]
  \item install a \TeX~distribution, 
  \item download a copy of this projects' files onto your local device, and then 
  \item compile \lstinline|CVexample.tex| to produce a PDF output.
\end{enumerate*}
This is a good litmus test to see if your software has been correctly installed. \par

For completion, those instructions are repeated here. \par

\subsection{Installation}
A \TeX~distribution, such as MikTeX, TeXLive or MacTeX, is a prerequisite for working with .tex files. They free and easily available for download online. \par

\TeX~files can be written using any editor, and compiled from the system shell with the command \lstinline|pdflatex <myfile>.tex|. However, for ease-of-use, you should probably install a dedicated graphical user interface for working with \TeX~files (e.g. TeXworks or TeXstudio). Instructions for downloading the appropriate software can be found by following the link: \url{https://www.latex-project.org/get/}. \par

\begin{instrct}
Install an appropriate TeX distribution and editor for your system. \par
\end{instrct}

Alternatively, Overleaf is an online LaTeX editor which doesn't require any local installation; but will ask you to register an account. \par

\subsection{Setup}
With the appropriate software now installed, download the package repository onto your local device. \par

\begin{instrct}
Navigate to \url{https://github.com/Pedro-h-mattos/LaTeXCVTemplate}, find and click on \lstinline|Download ZIP|.
\end{instrct}

Remember to unpackage the .zip folder, before continuing with these instructions. \par

\subsection{Your First Document!}
Compiling \lstinline|CVexample.tex| should output the document \lstinline|CVexample.pdf|. \par

You can always compile documents from the command line, by running the command \lstinline|pdflatex CVexample.tex|. \par

If you're using an \TeX~editor, first locate and open the file \lstinline|CVexample.tex| from within your project directory. Then compile it using the \lstinline|pdfLaTeX| option. \par

\begin{instrct}
Try compiling your first document; then, open the output file \lstinline|CVexample.pdf|.
\end{instrct}

