\section{Installation and Usage}

\subsection{Downloading Necessary Software}
A prerequisite for working with .tex files is to install a TeX distribution, such as MikTex, TexLive or MacTeX --- which enable your computer to use TeX code. They are free and easily available for download online. \par

TEX input files are plain text files, which can be written using any editor and compiled from the system shell. However for ease-of-use, you should install a dedicated graphical user interface for working with TeX files (e.g. TeXworks or TeXstudio). \par

\begin{instrct}
Install an appropriate TeX distribution and editor for your system. \par
\end{instrct}

Instructions for downloading the appropriate software can be found by following the link: \url{https://www.latex-project.org/get/}. \par

Alternatively, Overleaf is an online LaTeX editor which doesn't require any local installation. Although, does require setting up an account. \par

\subsection{Opening TeX Files}
Within a TeX editor, you are able to open and work with .tex files. \emph{CVexample.tex} is the main file for this project, which contains code that will output a document.

\begin{instrct}
Within your project directory, locate and open the file \emph{CVexample.tex}.
\end{instrct}

\subsection{Compiling TeX Files}
Compiling this code will output a PDF-format r\'esum\'e called \emph{CVexample.pdf}, located in your project directory. Depending on your editor, compiling may also automatically create a preview of your document. \par

Before compiling, take care to select the option \emph{pdfLaTeX}, otherwise it may not work. \par

\begin{instrct}
Try compiling your first .tex document. Then, locate and open your compiled pdf document within your project directory.
\end{instrct}

\subsection{Your First Document!}
Take a moment to examine your document. What do you notice? 

Although very neatly typset, your `personalised' r\'esum\'e lacks any personal information at all! \emph{CVexample.tex} is designed as a customisable template; the rest of this exercise will guide you in replacing each section of this document. \par

Now consider the structure of the template itself. How many sections does it contain? Can you identify any elements that are repeated? \par

The structure of a r\'esum\'e document is fairly standardized. Personal and/or contact information is usually placed in the document header and sections should correspond to a person's education, work experience, skills, publications, etc. \par

Here, the document header can be subdivided into three elements: a title, the author's name, and their personal and contact information; centered on the page. The latter of which is organised into two columns. \par 

The three sections, labelled \emph{Education/Qualifications}, \emph{Technical Skills} and \emph{Experience}, are always underlined and followed by subsections describing relevant experience. These subsections always compose two lines of text; one bold and larger, and the other italicized and smaller, that may or may not extend across the page. Optionally, a subsection can be followed by one or more bullet points. \par